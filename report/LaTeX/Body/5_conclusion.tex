\section{Conclusion and Future Work}

It is evident that number of WiFi devices detected at a location is indicative of the crowd strength. However, computing the exact number of people is a difficult task due to the limitations mentioned earlier. False detections due to randomized MAC addresses are a major challenge to overcome. Since newer operating systems use randomized MAC addresses for probes as well as associations, the access point spoofing approach mentioned in \textcite{brower_mac-based_nodate} is no longer accurate. The packet-linking technique proposed in \textcite{freudiger_how_2015} may work and needs to be investigated. But it may still not be possible to identify the original MAC address of a device.\\
Another interesting observation is that, at sites where majority of devices are expected to be connected to a WiFi access point, probe requests alone are not sufficient to measure the number of devices. Given the large amount of disk-space and processing time required for the packets, it is preferable to reduce the number of captured packets as much as possible. Thus in public places where no open WiFi networks are available, counting devices on the basis of only probe-requests, is a reasonable technique.